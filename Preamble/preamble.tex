% NOTE:
% THIS PREAMBLE IS OUT OF DATE! IT'S DIFFERENT TO THE OTHER ONES BECAUSE BEAMER.

% SECTION: PACKAGES
\usepackage[utf8]{inputenc}
% \usepackage[T1]{fontenc}
\usepackage{amssymb}\usepackage{amsmath}\usepackage{amsthm}\usepackage{mathtools}\usepackage{eucal}
\usepackage{tabularx}
\usepackage{nicefrac}\usepackage{extarrows}
\usepackage{graphicx}
\usepackage{wrapfig}\usepackage{subcaption}
\usepackage{eqparbox}
\usepackage{listings}\usepackage[numbered]{matlab-prettifier}
\usepackage[outputdir=build]{minted}
\usepackage{cprotect}
\usepackage{enumitem}
\usepackage{esint}
\usepackage{pdflscape}
\usepackage{mdframed}
\usepackage{tikz}\usetikzlibrary{automata,arrows,positioning,calc}
\usepackage{tikz-cd}
\usepackage[outline]{contour}
\usepackage{bookmark}
\usepackage{anyfontsize} % NOTE: Not sure why I need this

%%% INVERT PAGE COLOUR
\usepackage{xcolor}
\definecolor{LightGray}{gray}{0.9}
%\pagecolor[rgb]{0.1,0.1,0.1}
%\color[rgb]{1,1,1}

%%% CITATIONS & BIBLIOGRAPHY
\usepackage[backend=biber]{biblatex}
% \usepackage[backend=bibtex]{biblatex}
\addbibresource{bibliography.bib}

%%% NOT \usepackage{geometry}
%\usepackage{geometry}
%\geometry{
%  a4paper,
%  total={150mm,237mm},
%  left=30mm,
%  top=30mm,
%} % A4 dimensions: 210 x 297

% \usepackage{tikz-among-us}
\usepackage{tikz-among-us}

%%% \usepackage{hyperref}
% Load hyperref last!
\usepackage{hyperref}
\hypersetup{
  colorlinks = true,
  linkcolor = cyan,
  urlcolor = cyan,
}






%%%%%%%%%%%%%%%%%%%%%%%%%%%%%%%%%%%%%%%%%%%%%%%%%%%%%%
% Repeat table of contents on each section
%%%%%%%%%%%%%%%%%%%%%%%%%%%%%%%%%%%%%%%%%%%%%%%%%%%%%%
\AtBeginSection[]
{
  \begin{frame}
    \frametitle{Outline}
    \tableofcontents[currentsection]
  \end{frame}
}




%%%%%%%%%%%%%%%%%%%%%%%%%%%%%%%%%%%%%%%%%%%%%%%%%%%%%%
% Util
%%%%%%%%%%%%%%%%%%%%%%%%%%%%%%%%%%%%%%%%%%%%%%%%%%%%%%
\newcommand{\comment}[1]{}
\newcommand{\attentive}[1]{\textcolor{pink}{#1}}






%%%%%%%%%%%%%%%%%%%%%%%%%%%%%%%%%%%%%%%%%%%%%%%%%%%%%%
% Double aligned inequalities
%%%%%%%%%%%%%%%%%%%%%%%%%%%%%%%%%%%%%%%%%%%%%%%%%%%%%%
% https://tex.stackexchange.com/a/34412/5764
\makeatletter
\NewDocumentCommand{\eqmathbox}{o O{c} m}{%
  \IfValueTF{#1}
    {\def\eqmathbox@##1##2{\eqmakebox[#1][#2]{$##1##2$}}}
    {\def\eqmathbox@##1##2{\eqmakebox{$##1##2$}}}
  \mathpalette\eqmathbox@{#3}
}
\makeatother

\newcounter{twoineq}
\newcommand{\fiveline}[5]{
  #1 
  &#2\; 
  &\eqmathbox[\thetwoineq]{#3} 
  &#4 
  #5
}



%%%%%%%%%%%%%%%%%%%%%%%%%%%%%%%%%%%%%%%%%%%%%%%%%%%%%%
% Matrix extension
%%%%%%%%%%%%%%%%%%%%%%%%%%%%%%%%%%%%%%%%%%%%%%%%%%%%%%
\makeatletter
\renewcommand*\env@matrix[1][*\c@MaxMatrixCols c]{%
  \hskip -\arraycolsep
  \let\@ifnextchar\new@ifnextchar
  \array{#1}}
\makeatother



%%%%%%%%%%%%%%%%%%%%%%%%%%%%%%%%%%%%%%%%%%%%%%%%%%%%%%
% Standard symbols
%%%%%%%%%%%%%%%%%%%%%%%%%%%%%%%%%%%%%%%%%%%%%%%%%%%%%%
\newcommand*\df{\mathop{}\!\mathrm{d}}
\newcommand{\e}{\mathrm{e}}

\newcommand{\deriv}[1]{\frac{\df}{\df #1}}
\newcommand{\derivup}[2]{\frac{\df #1}{\df #2}}
\newcommand{\pderiv}[1]{\frac{\partial}{\partial #1}}
\newcommand{\pderivup}[2]{\frac{\partial #1}{\partial #2}}


\newcommand{\powset}[1]{\mathcal{P}\left(#1\right)}
\newcommand{\pre}{\leftarrow}

\newcommand{\one}{\mathbf{1}}
\newcommand{\zero}{\mathbf{0}}

\newcommand{\restrict}[2]{\left. #1 \right\rvert_{#2}}



%%%%%%%%%%%%%%%%%%%%%%%%%%%%%%%%%%%%%%%%%%%%%%%%%%%%%%
% \set, my favourite <3
%%%%%%%%%%%%%%%%%%%%%%%%%%%%%%%%%%%%%%%%%%%%%%%%%%%%%%
\DeclarePairedDelimiterX\set[1]\lbrace\rbrace{\def\given{\;\delimsize\vert\;}#1}
% Adaptation to parentheses, for conditional probabilities
\DeclarePairedDelimiterX\cond[1]\lparen\rparen{\def\given{\;\delimsize\vert\;}#1}



%%%%%%%%%%%%%%%%%%%%%%%%%%%%%%%%%%%%%%%%%%%%%%%%%%%%%%
% Assignment-specific
%%%%%%%%%%%%%%%%%%%%%%%%%%%%%%%%%%%%%%%%%%%%%%%%%%%%%%
\newcommand{\q}[1]{\section*{Question #1} \stepcounter{section}}
\newcommand{\sq}[2]{\subsection*{Question #1#2} \stepcounter{subsection}}
\newcommand{\ssq}[3]{\subsubsection*{Question #1#2 part #3} \stepcounter{subsubsection}}

\newcommand{\setq}[3]{
  \setcounter{section}{#1}
  \setcounter{subsection}{#2}
  \setcounter{subsubsection}{#3}
}
\newcommand{\incq}{
  \stepcounter{section}
}
\newcommand{\incsq}{
  \stepcounter{subsection}
}
\newcommand{\incssq}{
  \stepcounter{subsubsection}
}

\newcommand{\tq}{
  \section*{
    Question \arabic{section}
  }
}
\newcommand{\tsq}{
  \section*{
    Question \arabic{section}\alph{subsection}
  }
}
\newcommand{\tssq}{
  \section*{
    Question \arabic{section}\alph{subsection} part \roman{subsubsection}
  }
}

\newcommand{\pf}{\par\textbf{Proof.}}
\newcommand{\sol}{\par\textbf{Solution.}}



%%%%%%%%%%%%%%%%%%%%%%%%%%%%%%%%%%%%%%%%%%%%%%%%%%%%%%
% Math operators
%%%%%%%%%%%%%%%%%%%%%%%%%%%%%%%%%%%%%%%%%%%%%%%%%%%%%%
% Set theory
\DeclareMathOperator{\card}{card}
\newcommand{\img}{}
\newcommand{\pimg}{{-1}}

% Category theory
\DeclareMathOperator{\Set}{\mathbf{Set}}
\DeclareMathOperator{\Grp}{\mathbf{Grp}}
\DeclareMathOperator{\Vect}{\mathbf{Vect}}
\DeclareMathOperator{\NormedSpace}{\mathbf{NS}}

% Linear algebra
\DeclareMathOperator{\spans}{span}
\DeclareMathOperator{\Lin}{\mathcal{L}}
\DeclareMathOperator{\rank}{rank}
\DeclareMathOperator{\tr}{tr}

% Matrix-vector
\renewcommand{\vec}[1]{\mathbf{#1}}
\newcommand{\transpose}{{\top}}

% Group theory
\DeclareMathOperator{\im}{im}
\DeclareMathOperator{\rng}{rng}

\DeclareMathOperator{\GL}{GL}
\DeclareMathOperator{\SL}{SL}

\DeclareMathOperator{\Stab}{Stab}
\DeclareMathOperator{\Orb}{\mathcal{O}}

% Common functions
\DeclareMathOperator{\sgn}{sgn}

% Real analysis

% Complex analysis
\let\Re\relax\DeclareMathOperator{\Re}{Re}
\let\Im\relax\DeclareMathOperator{\Im}{Im}

% Topology

% Metric space topology
\DeclareMathOperator{\diam}{diam}

% Measure theory
\newcommand{\Leb}{\mathcal{L}}
\newcommand{\Haus}{\mathcal{H}}

% Fractal geometry
\DeclareMathOperator{\dimHaus}{\dim_{\mathcal{H}}}

% Vector calculus
\DeclareMathOperator{\divs}{div}
\renewcommand{\div}{\divs}
\DeclareMathOperator{\curl}{curl}

% Probability theory
\DeclareMathOperator{\PP}{\mathbb{P}}
\DeclareMathOperator{\EE}{\mathbb{E}}
\DeclareMathOperator{\Var}{\mathbb{V}ar}
\DeclareMathOperator{\Cov}{\mathbb{C}ov}
\newcommand{\widesim}[2][1.5]{
  \mathrel{\overset{#2}{\scalebox{#1}[1]{$\sim$}}}
}
\newcommand{\simiid}{\widesim{\phantom{\text{.}}\text{iid}\phantom{\text{.}}}}

\DeclareMathOperator{\Ber}{Ber}
\DeclareMathOperator{\Bin}{Bin}
\DeclareMathOperator{\Geom}{Geom}
\DeclareMathOperator{\Poi}{Poi}
\DeclareMathOperator{\Hyp}{Hyp}
\DeclareMathOperator{\Unif}{\mathcal{U}}
\DeclareMathOperator{\Exp}{Exp}
\DeclareMathOperator{\Norm}{\mathcal{N}}
\DeclareMathOperator{\Gam}{Gamma}

% ??
\DeclareMathOperator{\meas}{meas}



%%%%%%%%%%%%%%%%%%%%%%%%%%%%%%%%%%%%%%%%%%%%%%%%%%%%%%
% THEOREM STYLE
%%%%%%%%%%%%%%%%%%%%%%%%%%%%%%%%%%%%%%%%%%%%%%%%%%%%%%
\newtheoremstyle{break}
  {\topsep}{\topsep}%
  {\itshape}{}%
  {\bfseries}{}%
  {\newline}{}%
\theoremstyle{break}



%%%%%%%%%%%%%%%%%%%%%%%%%%%%%%%%%%%%%%%%%%%%%%%%%%%%%%
% Standard theorem styles
%%%%%%%%%%%%%%%%%%%%%%%%%%%%%%%%%%%%%%%%%%%%%%%%%%%%%%
\theoremstyle{break}
%\newtheorem{theorem}{Theorem}[subsection]
%\newtheorem{proposition}[theorem]{Proposition}
%\newtheorem{lemma}[theorem]{Lemma}
%\newtheorem{corollary}[theorem]{Corollary}

%\theoremstyle{remark}
%\newtheorem{claim}[theorem]{Claim}



%%%%%%%%%%%%%%%%%%%%%%%%%%%%%%%%%%%%%%%%%%%%%%%%%%%%%%
% QED symbol
%%%%%%%%%%%%%%%%%%%%%%%%%%%%%%%%%%%%%%%%%%%%%%%%%%%%%%
\newcommand{\bsq}{\hfill$\blacksquare$}
% Current QED symbol is:
%   An amogus from \usepackage{tikz-among-us}
\renewcommand{\qedsymbol}{
  \begin{tikzpicture}
    \amongUsIII[xscale = -0.04, yscale = 0.04]{pink!50!white}{white}
  \end{tikzpicture}
}



%%%%%%%%%%%%%%%%%%%%%%%%%%%%%%%%%%%%%%%%%%%%%%%%%%%%%%
% Misc
%%%%%%%%%%%%%%%%%%%%%%%%%%%%%%%%%%%%%%%%%%%%%%%%%%%%%%
\newcommand{\mcom}[1]{\text{\emph{#1}}}
\newcommand{\smallclaim}{\par\textbf{Claim.}}
\newcommand{\ab}[1]{{#1}^{\text{ab}}}
\newcommand{\Sierpinski}{Sierpi\'nski}
\DeclareMathOperator{\Size}{Size}
\DeclareMathOperator{\vol}{vol}



%%%%%%%%%%%%%%%%%%%%%%%%%%%%%%%%%%%%%%%%%%%%%%%%%%%%%%
% Actually just memes
%%%%%%%%%%%%%%%%%%%%%%%%%%%%%%%%%%%%%%%%%%%%%%%%%%%%%%
\newcommand{\brad}{\nabla}






%%%%%%%%%%%%%%%%%%%%%%%%%%%%%%%%%%%%%%%%%%%%%%%%%%%%%%
% No idea what this is, but I'm afraid to delete it
%%%%%%%%%%%%%%%%%%%%%%%%%%%%%%%%%%%%%%%%%%%%%%%%%%%%%%
\newcommand*\centermathcell[1]{\omit\hfil$\displaystyle#1$\hfil\ignorespaces}
\newlength\wantedwidth
\newcommand{\fakewidth}[2]{%
  \settowidth{\wantedwidth}{\ensuremath{#2}}%
  \makebox[\wantedwidth]{\ensuremath{#1}}%
}

%%%%%%%%%%%%%%%%%%%%%%%%%%%%%%%%%%%%%%%%%%%%%%%%%%%%%%
% Obligatory
%%%%%%%%%%%%%%%%%%%%%%%%%%%%%%%%%%%%%%%%%%%%%%%%%%%%%%
\allowdisplaybreaks






%%%%%%%%%%%%%%%%%%%%%%%%%%%%%%%%%%%%%%%%%%%%%%%%%%%%%%
% UNUSED
%%%%%%%%%%%%%%%%%%%%%%%%%%%%%%%%%%%%%%%%%%%%%%%%%%%%%%
%\usepackage{apacite}
%\bibliographystyle{apacite} %This seems to work

%\newtheorem{theorem}{Theorem}[section]
%\newtheorem{lemma}{Lemma}[section]
%\newtheorem{corollary}{Corollary}[section]
%\newtheorem{proposition}{Proposition}[section]
%\newtheorem{claim}{Claim}[section]

%\newcommand{\mc}[2]{
%    \left( \begin{array}{c} 
%        #1 \\
%        \alpha_0, \dots, \alpha_{#2}
%    \end{array} \right)
%}
%\newcommand{\mcalpha}[1]{
%    \left( \begin{array}{c} 
%        #1 \\
%        \alpha
%    \end{array} \right)
%}
%\newcommand{\mcoeff}{
%    \left( \begin{array}{c} 
%        n \\
%        \alpha_1, \alpha_2, \dots, \alpha_{C}
%    \end{array} \right)
%}
%\newcommand{\smcoeff}{
%    \left( \begin{array}{c} 
%        n \\
%        \alpha_1, \dots, \alpha_{C}
%    \end{array} \right)
%}
%\newcommand{\imcoeff}{
%    \left( \begin{array}{c} 
%        n \\
%        \alpha
%    \end{array} \right)
%}
%\newcommand{\Mod}[1]{\ (\mathrm{mod}\ #1)}
%\newcommand{\modbounds}{
%    \begin{array}{c}
%        \alpha_1 + \cdots + \alpha_{C} = \, n \\
%        g_1^{\alpha_1} \cdots g_C^{\alpha_C} = g_\omega
%    \end{array}
%}
%\newcommand{\imcoeffN}{\left( \begin{array}{c} 
%        N \\
%        \alpha
%    \end{array} \right)
%}
%\newcommand{\ACGR}{A$\mathbb{C}$GR }
%\newcommand{\ACGRs}{A$\mathbb{C}$GRs }
%\newcommand{\acgreq}{\left| \gring{\mathbb{C}}{G} \right|}
%\newcommand{\cgring}[1]{\gring{\mathbb{C}}{C_{#1}}}
%\newcommand{\ccgring}[1]{\gring{\mathbb{C}}{#1}}
%\newcommand{\diam}[1]{\mathrm{diam}\left(#1\right)}
%\newcommand{\avg}[1]{\underset{#1}{\mathrm{avg}}}
%\newcommand*\dm{\df \mu}
%\newcommand*\dn{\df \nu}
%\newcommand{\EC}[1]{\mathbb{E}\left(C_{#1}\right)}
%\newcommand{\amax}{\alpha_{\mathrm{max}}}
%\newcommand{\amin}{\alpha_{\mathrm{min}}}
%\newcommand{\Rer}{\operatorname{Re}}
%\newcommand{\Imr}{\operatorname{Im}}
%\newcommand{\smax}{\sigma_{\mathrm{max}}}
%\newcommand{\smin}{\sigma_{\mathrm{min}}}
%\newcommand{\dom}[1]{\mathrm{dom}_{#1}}
%\newcommand{\dommu}{\dom{\mu}}
%\newcommand{\domleb}{\dom{\lambda}}






%%%%%%%%%%%%%%%%%%%%%%%%%%%%%%%%%%%%%%%%%%%%%%%%%%%%%%
% Notes to self
%%%%%%%%%%%%%%%%%%%%%%%%%%%%%%%%%%%%%%%%%%%%%%%%%%%%%%
% NOTE TO SELF: SYNTAX FOR RECALLING THEOREMS
%\begingroup
%\def\thetheorem{\ref{thm_TheoremName}}
%\begin{theorem} [Displayed text]
%    Theorem text
%\end{theorem}
%\addtocounter{theorem}{-1}
%\endgroup
