% Agenda:
%   [x] Define $\coN$, $\pred$
%   [x] Partial functions
%   [x] Corecursion principle ($\wait$)
%   [x] Examples of corecursion

\begin{frame}{The Conatural Numbers} % .;

  \begin{itemize}[nosep]
    \item[$\bullet$] $\eqmathbox[conat-n][l]{\N} \to \eqmathbox[conat-rec][l]{\text{Recursion}} \to \eqmathbox[conat-ind][l]{\text{Induction}}$
    \item[$\bullet$] $\eqmathbox[conat-n]{\coN} \to \eqmathbox[conat-rec]{\text{Corecursion}} \to \eqmathbox[conat-ind]{\text{Coinduction}}$
  \end{itemize}

  \pause

  \begin{block}{Definition: Conatural numbers, Predecessor}

    \par The set of \strong{conatural numbers} is $\coN := \N \squ \Infty$.
    \pause
    \par The \strong{predecessor} operation is the bijection
    \begin{align*}
      \pred : \coN
      &\longrightarrow \No \squ \coN
      \\
      0
      &\longmapsto \no
      \\
      \Succ(n)
      &\longmapsto n
      \\
      \infty
      &\longmapsto \infty
    \end{align*}
  \end{block}

  \pause

  \par $\N$ comes with $\No \squ \N \xrightarrow{(0, \Succ)} \N$.
  \par $\coN$ comes with $\coN \xrightarrow{\pred} \No \squ \coN$.

\end{frame}

\begin{frame}{Partial Functions} % .;
  
  \begin{block}{Definition: Partial function}

    \par A \strong{partial function} $X \nrightarrow Y$ is a function $X \to \No \squ Y$.

  \end{block}

  \par Examples:\pause
  \begin{itemize}
    \item[$\bullet$] {
      $f : x \mapsto 1 / x : \mathbb{R} \nto \mathbb{R}$;
      \newline
      $f(0) = \no$.
      \newline
    }\pause
    \item[$\bullet$] {
      ``Get input from user'' $: \texttt{ComputerState} \nto \texttt{String}$;
      \newline
      Fails when the user doesn't give input.
      \newline
    }\pause
    \item[$\bullet$] {
      $\pred : \coN \nto \coN$;
      \newline
      $\pred(0)$ fails.
    }
  \end{itemize}

\end{frame}

\begin{frame}{Corecursion into $\coN$} % .;

  \begin{block}{Corecursion principle}
    \par Let $f : X \nto X$ be a partial function. Construct $u : X \to \coN$ by{\pause}
    \begin{itemize}[nosep]
      \item[$\bullet$] $u(x) = 0 \iff f(x) = \no${\pause}
      \item[$\bullet$] $u(x) = 1 \iff f(x) \neq \no \tand f(f(x)) = \no${\pause}
      \item[$\bullet$] $u(x) = 2 \iff f(x) \neq \no \tand f(f(x)) \neq \no \tand f^3 (x) = \no$
      \item[$\bullet$] etc.{\pause}
      \item[$\bullet$] $u(x) = \infty \iff f(x) \neq \no \tand f^2 (x) \neq \no \tand \cdots${\pause}
    \end{itemize}
    \par Write $u := \wait (f)$.
  \end{block}

  \pause

  \begin{block}{Lemma: Pred-wait}
    
    \par Let $f : X \nto X$ be a partial function and $x \in X$.
    \begin{itemize}[nosep]
      \item[$\bullet$] If $f(x) = \no$, then $\pred(\wait(f)(x)) = \no$.
      \item[$\bullet$] If $f(x) = x' \neq \no$, then $\pred(\wait(f)(x)) = \wait(f)(x')$.
    \end{itemize}

  \end{block}

  \par \textcolor{gray}{Abstract nonsense: $\pred$ is the terminal partial endo-function.}

\end{frame}

\begin{frame}{Corecursion into $\coN$} % .;
  
  \par Example:
  \begin{itemize}
    \item[$\bullet$] {
      Let $\textcolor{pink}{c : \mathbb{Z}_{> 0} \nto \mathbb{Z}_{> 0}}$ by
      \begin{align*}
        c(1) &:= \no;
        & c(n) &:= n / 2 \text{ if } n \text{ is even};
        & c(n) &:= 3n + 1 \text{ otherwise}
      \end{align*}
      \vspace{-\baselineskip}
    }\pause
    \item[$\bullet$] {
      $\wait(c)(x)$ is the length of $x \xmapsto{c} \cdots \xmapsto{c} 1$, possibly $\infty$.
      \newline
    }\pause
    \item[$\bullet$] {
      E.g. $5 \xmapsto{c} 16 \xmapsto{c} 8 \xmapsto{c} 4 \xmapsto{c} 2 \xmapsto{c} 1 \xmapsto{c} \no$, so $\wait(c)(5) = 5$.
      \newline
    }\pause
    \item[$\bullet$] {
      The \textcolor{pink}{Collatz conjecture} asks whether $\wait(c)(n) \neq \infty$ for all $n \in \mathbb{Z}_{> 0}$.
    }
  \end{itemize}

\end{frame}

\begin{frame}{Corecursion into $\coN$} % .;
  
  \begin{block}{Definition: Addition in $\coN$}
    
    \par The function $(\blank + \blank) : \coN \x \coN \to \coN$ is defined by
    \begin{align*}
      (\blank + \blank) := \wait(f)
    \end{align*}
    where $f : \coN \x \coN \nto \coN \x \coN$ is given by
    \begin{align*}
      f : (x, y)
      &\mapsto
      \begin{cases}
        \no & \text{if } \pred(x) = \pred(y) = \no \\
        (x, y') &\text{if } \pred(x) = \no \tand \pred(y) = y' \neq \no \\
        (x', y) &\text{if } \pred(x) = x' \neq \no
      \end{cases}
    \end{align*}

  \end{block}

  \pause

  \par By the pred-wait lemma,{\pause}
  \begin{itemize}[nosep]
    \item[$\bullet$] {
      $\pred (0 + 0) = \no$, so $0 + 0 = 0$;
    }\pause
    \item[$\bullet$] {
      $\pred(0 + y) = \pred(y)$, so $0 + y = y$;
    }\pause
    \item[$\bullet$] {
      $\pred(x + y) = \pred(x) + y$ whenever $\pred(x) \neq \no$.
    }
  \end{itemize}

\end{frame}
