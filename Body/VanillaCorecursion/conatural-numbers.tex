% Agenda:
%   [x] Define $\coN$, $\pred$
%   [x] Partial functions
%   [x] Corecursion principle ($\wait$)
%   [ ] Examples of corecursion

\begin{frame}{The Conatural Numbers} % .;

  \begin{itemize}[nosep]
    \item[$\bullet$] $\eqmathbox[conat-n][l]{\N} \to \eqmathbox[conat-rec][l]{\text{Recursion}} \to \eqmathbox[conat-ind][l]{\text{Induction}}$
    \item[$\bullet$] $\eqmathbox[conat-n]{\coN} \to \eqmathbox[conat-rec]{\text{Corecursion}} \to \eqmathbox[conat-ind]{\text{Coinduction}}$
  \end{itemize}

  \pause

  \begin{block}{Definition: Conatural numbers, Predecessor}

    \par The set of \strong{conatural numbers} is $\coN := \N \squ \Infty$.
    \par The \strong{predecessor} operation is
    \begin{align*}
      \pred : \coN
      &\longrightarrow \No \squ \coN
      \\
      0
      &\longmapsto \no
      \\
      \Succ(n)
      &\longmapsto n
      \\
      \infty
      &\longmapsto \infty
    \end{align*}
  \end{block}

  \pause

  \par $\N$ comes with $\No \squ \N \xrightarrow{(0, \Succ)} \N$.
  \par $\coN$ comes with $\coN \xrightarrow{\pred} \No \squ \N$.

\end{frame}

\begin{frame}{Partial Functions} % .;
  
  \begin{block}{Definition: Partial function}

    \par A \strong{partial function} $X \nrightarrow Y$ is a function $X \to \No \squ Y$.

  \end{block}

  \par Examples:\pause
  \begin{itemize}
    \item[$\bullet$] {
      $f : x \mapsto 1 / x : \mathbb{R} \nto \mathbb{R}$;
      \newline
      $f(0) = \no$.
      \newline
    }\pause
    \item[$\bullet$] {
      ``Get input from user'' $: \texttt{ComputerState} \nto \texttt{String}$;
      \newline
      Fails when the user doesn't give input.
      \newline
    }\pause
    \item[$\bullet$] {
      $\pred : \coN \nto \coN$;
      \newline
      $\pred(0)$ fails.
    }
  \end{itemize}

\end{frame}

\begin{frame}{Corecursion into $\coN$} % .;

  \begin{block}{Corecursion principle}
    \par Let $f : X \nto X$ be a partial function. Construct $u : X \to \coN$ by{\pause}
    \begin{itemize}[nosep]
      \item[$\bullet$] $u(x) = 0 \iff f(x) = \no${\pause}
      \item[$\bullet$] $u(x) = 1 \iff f(x) \neq \no \tand f(f(x)) = \no${\pause}
      \item[$\bullet$] $u(x) = 2 \iff f(x) \neq \no \tand f(f(x)) \neq \no \tand f^3 (x) = \no$
      \item[$\bullet$] etc.{\pause}
      \item[$\bullet$] $u(x) = \infty \iff f(x) \neq \no \tand f^2 (x) \neq \no \tand \cdots${\pause}
    \end{itemize}
    \par Write $u := \wait (f)$.
  \end{block}

  \pause

  \begin{block}{Lemma: Pred-wait}
    
    \par Let $f : X \nto X$ be a partial function and $x \in X$.
    \begin{itemize}[nosep]
      \item[$\bullet$] If $f(x) = \no$, then $\pred(\wait(f)(x)) = \no$.
      \item[$\bullet$] If $f(x) = x' \neq \no$, then $\pred(\wait(f)(x)) = \wait(f)(x')$.
    \end{itemize}

  \end{block}

  \par \textcolor{gray}{Abstract nonsense: $\pred$ is the terminal partial endo-function.}

\end{frame}

\begin{frame}{Corecursion into $\coN$: $\tail$} % .;
  
  \par Example:
  \begin{itemize}
    \item[$\bullet$] {
      Let $\textcolor{pink}{\coList(X)}$ be the set of (finite or infinite) sequences on $X$.
    }\pause
    \item[$\bullet$] {
      Let $\textcolor{pink}{\tail : \coList(X) \nto \coList(X)}$, with
      \begin{align*}
        \tail([x_0, x_1, \dots]) &= [x_1, \dots]
        &\tail([]) &= \no
      \end{align*}
    }\pause
    \item[$\bullet$] {
      Then, $\wait(\tail)(x)$ is the length of $x$, possibly $= \infty$.
    }\pause
    \item[$\bullet$] {
      The lemma says
      \begin{align*}
        \pred(\wait(\tail)([])) &= \no \\
        \pred(\wait(\tail)([x_0, x_1, \dots])) &= \wait(\tail)([x_1, \dots])
      \end{align*}
    }
  \end{itemize}

\end{frame}

\begin{frame}{Corecursion into $\coN$: addition} % .;
  
  \par 

\end{frame}
