% Agenda:
%   [x] Endofunctor
%   [ ] General recursion principle
%   [ ] General simple coinduction

\begin{frame}{All the categories} % .;
  
  \par All of the inductive \texttt{data} types are \strongPink{initial algebras}.

  \begin{block}{Initial algebras}
    
    \par Let $T : \mathcal{C} \to \mathcal{C}$ be an endofunctor.
    \vspace{0.5 \baselineskip}
    \pause
    \par An \strong{algebra} for $T$ is an arrow $a \xrightarrow{\alpha} T(a)$ in $\mathcal{C}$.
    \vspace{0.5 \baselineskip}
    \pause
    \par A \strong{morphism} from $a \xrightarrow{\alpha} T(a)$ to $b \xrightarrow{\beta} T(b)$ is an arrow $f : a \to b$ in $\mathcal{C}$ such that this commutes:
    \begin{align*}
      \begin{tikzcd}[cramped, ampersand replacement = \;]
        a \arrow[r, "\alpha"] \arrow[d, "f"'] \; T(a) \arrow[d, "T(f)"] \\
        b \arrow[r, "\beta"'] \; T(b)
      \end{tikzcd}
    \end{align*}
    \vspace{-0.5 \baselineskip}
    \pause
    \par With identity and composite arrows as in $\mathcal{C}$, these form a category.
    \vspace{0.5 \baselineskip}
    \pause
    \par An \strong{initial algebra} for $T$ is an initial object in this category.
    
  \end{block}

\end{frame}

\begin{frame}{Recursion/Induction} % .;

  \par All of the inductive \texttt{data} types are \strongPink{initial algebras}.
  
  \pause
  
  \begin{block}{Recursion}

    \par An \strong{initial algebra} for $T$ is an initial object $i \xrightarrow{\cons} T(i)$ in the category of $T$-algebras.
    \vspace{0.5 \baselineskip}
    \pause
    \par The \strong{recursion principle} is its universal property.
    
  \end{block}

\end{frame}

\begin{frame}{Recursion/Induction} % .;
  
  \begin{block}{Induction}
    
    \par Let $i \xrightarrow{\cons} T(i)$ be an initial algebra.
    \par Let $f, g : i \to a$ in $\mathcal{C}$.
    \par Suppose the equaliser $\iota : e \rightarrowtail i$ of $f$ and $g$ exists in $\mathcal{C}$, and that $f \circ \cons \circ T(\iota) = g \circ \cons \circ T(\iota)$:
    \vspace{-0.5 \baselineskip}
    \begin{align*}
      % NOTE: Vertical version of diagram
      \begin{tikzcd}[cramped, ampersand replacement = \;]
        % Top row
          T(e)
            \arrow[r, dashed]
            \arrow[d, "T(\iota)"']
          \;
          e
            \arrow[d, "\iota", tail]
        \\
        % Middle row
          T(i)
            \arrow[r, "\cons"]
          \;
          i
            \arrow[d, "f"', shift right]
            \arrow[d, "g", shift left]
        \\
        % Bottom row
          % (blank)
          \;
          a
      \end{tikzcd}
    \end{align*}
    \par Then, $f = g$.

  \end{block}

\end{frame}

\begin{frame}{Corecursion/Coinduction} % .;

  \par \strong{Turn all the arrows around.}

  \pause
  
  \begin{block}{Coinduction}
    
    \par Let $T(j) \xrightarrow{\des} j$ be an initial algebra.
    \par Let $f, g : a \to j$ in $\mathcal{C}$.
    \par Suppose the coequaliser $\pi : j \twoheadrightarrow e$ of $f$ and $g$ exists in $\mathcal{C}$, and that this commutes:
    \vspace{-0.5 \baselineskip}
    \begin{align*}
      % NOTE: Vertical version of diagram
      \begin{tikzcd}[cramped, ampersand replacement = \;]
        % Top row
          a
            \arrow[d, "f"', shift right]
            \arrow[d, "g", shift left]
          \;
          % (blank)
        \\
        % Middle row
          j
            \arrow[r, "\des"]
            \arrow[d, "\pi"', two heads]
          \;
          T(j)
            \arrow[d, "T(\pi)"]
        \\
        % Bottom row
          e
            \arrow[r, dashed]
          \;
          T(e)
      \end{tikzcd}
    \end{align*}
    \par Then, $f = g$.

  \end{block}

\end{frame}
