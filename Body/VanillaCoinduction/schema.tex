% Agenda:
%   [x] Thm. Simple coinduction
%     [x] Proof sketch? --> include j.i.c

\begin{frame}[t]{Coinduction into $\coN$} % .;

  \par Given $f, g : X \to Y$, let $\quot{Y}{f = g}$ be the quotient of $Y$ by the smallest equivalence relation $\sim$ with $f(x) \sim g(x)$ for each $x \in X$.
  \vspace{0.5 \baselineskip}
  \pause
  \par So, $[f(x)] = [g(x)]$ in $\quot{Y}{f = g}$.

  \pause
  
  \begin{block}{Theorem: Simple coinduction}
    \par Let $\lhs, \rhs : X \to \coN$ be two functions into $\coN$.
    \par Let $[\blank] : \coN \to \quot{\coN}{\lhs = \rhs}$ be the quotient map.
    \par If
    \begin{align*}
      \forall x \in X,\quad
      [\pred (\lhs (x))]
      &= [\pred (\rhs (x))]
    \end{align*}
    (with $[\no] := \no$){\pause}, then $\lhs = \rhs$.
  \end{block}

\end{frame}

\begin{frame}[t]{Coinduction into $\coN$: proof sketch} % .;
  
  \begin{block}{Theorem: Simple coinduction}
    \par If $\forall x \in X,\,
      [\pred (\lhs (x))]
      = [\pred (\rhs (x))]
      \text{ in } \quot{\coN}{\lhs = \rhs}
    $, then $\lhs = \rhs$.
  \end{block}
  \par\strong{Proof sketch.}
  \par It follows that $[\pred^n (\lhs(x))] = [\pred^n (\rhs(x))]$ for all $n$ (with $\pred(\no) := \no$).
  \pause
  \begin{itemize}
    \item[$\bullet$] {
      If $\lhs(x) = n \in \N$, then
      \begin{align*}
        [\pred^{n + 1} (\rhs(x))]
        = [\pred^{n + 1} (\lhs(x))]
        = [\no]
        = \no
      \end{align*}
      so $\pred^{n + 1} (\rhs(x)) = \no$, and hence $\rhs(x) = n = \lhs(x)$.
    }\pause
    \item[$\bullet$] {
      If $\rhs(x) \in \N$, do the same thing.
    }\pause
    \item[$\bullet$] {
      The only remaining case is $\lhs(x) = \infty = \rhs(x)$.
    }\pause
  \end{itemize}
  \par Therefore, $\lhs = \rhs$. \qed

\end{frame}
